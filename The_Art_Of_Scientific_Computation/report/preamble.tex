%---------------------------PREAMBLE
\documentclass[a4paper,10.5pt,oneside]{report} %book format on A4 paper with size 12 font
\usepackage[margin=2.0cm]{geometry}
\usepackage{amsmath,amssymb}            %a fairly generic package for mathematical output
%\usepackage{a4wide}                    %stretch the page margins
\usepackage{fancyhdr}                   %to customise the header and footer
%\usepackage{pxfonts}                   %use my usual font style
%\usepackage{abstract}                  %putting an abstract somewhere
\usepackage[usenames,dvipsnames]{color} %needs no justification
%\usepackage{titlesec}                  %change the properties of titles
\usepackage[titletoc]{appendix}         %customise appendices
\usepackage{pdfpages}                   %insert other pdfs into the document
\usepackage{enumerate}                  %gives the enumerate environment an optional argument
\usepackage[symbol*]{footmisc}          %for customising footnotes
\usepackage{soul}                       %provides hyphenatable letter spacing, striking and so forth
\usepackage{appendix}                   %appendices
\setcounter{tocdepth}{1}                %level of detail in the table of contents

%-----------MATH, TABULAR AND GRAPHICS PACKAGES
\usepackage{amsfonts}                   %extra fonts in math mode
\usepackage{slashed}                    %for Feynman slash notation
\usepackage{bm}                         %for bold math symbols
\usepackage{array}                      %used to extend the array and tabular environments
\usepackage{multirow}                   %make a table entry that will span multiple rows
\usepackage{graphicx}           %optional arguments for the \includegraphics command
\usepackage[hang,small,bf]{caption}     %captions under figures
\usepackage{subcaption}                 %subcaptions
\usepackage{wrapfig}                    %makes for nicer captions under figures
\usepackage{sidecap}                    %side captions
\usepackage[cdot,mediumqspace,amssymb]{SIunits}
                                        %include SI units package

%-----------LISTINGS PACKAGE AND CODE INCLUSION
\usepackage{listings}        
\usepackage{verbatim}

\lstset{language=[77]Fortran,
  basicstyle  =\ttfamily,
  keywordstyle=\color{red},
  commentstyle=\color{Green}
}

%-----------COUNTERS, HEADINGS AND REMARK ENVIRONMENTS
\newcommand{\note}[1]{\marginpar{\textcolor{red}{#1}}}

\pagestyle{plain}                       %puts headings on each page, subject to doc style
%\headheight 15pt                       %distance between header and top margin
%\lhead{}                               %header is on the left page
\numberwithin{equation}{chapter}        %equations look like (a.b)
%\numberwithin{equation}{section}       %equations look like (a.b.c)
%\numberwithin{figure}{section}         %figures look like (a.b.c)
%\setcounter{chapter}{-1}               %change the chapter seed
\let\endtocpage\relax                   %no new page after table of contents
\usepackage{tocloft}

%-----------MATH MODE SHORTCUTS
\newcommand{\vect}[1]{\boldsymbol{#1}}                   %bold vector
\newcommand{\mtx}[1]{\smash{\underline{\underline{\mathbf{#1}}}}} %matrix
\let\vaccent=\v                                          %rename builtin command \v{} to \vaccent{}
\renewcommand{\v}[1]{\ensuremath{\mathbf{#1}}}           %for vectors
\newcommand{\gv}[1]{\ensuremath{\mbox{\boldmath$ #1 $}}} %for vectors of Greek letters
\newcommand{\uv}[1]{\ensuremath{\mathbf{\hat{#1}}}}      %for unit vector
\newcommand{\abs}[1]{\left| #1 \right|}                  %for absolute value
\newcommand{\avg}[1]{\left< #1 \right>}                  %for average
\let\underdot=\d                                         %rename builtin command \d{} to \underdot{}
\newcommand{\deriv}[2]{\frac{\mathrm{d} #1}{\mathrm{d} #2}}                %for derivatives
\newcommand{\dd}[2]{\frac{\mathrm{d}^2 #1}{\mathrm{d} #2^2}}               %for double derivatives
\newcommand{\pd}[2]{\frac{\partial #1}{\partial #2}}     %for partial derivatives
\newcommand{\pdd}[2]{\frac{\partial^2 #1}{\partial #2^2}}%for double partial derivatives
\newcommand{\ket}[1]{| #1 \rangle}                       %for Dirac bras
\newcommand{\bra}[1]{\langle #1 |}                       %for Dirac kets
\newcommand{\braket}[2]{\langle #1 \vphantom{#2} | #2 \vphantom{#1} \rangle}                                   
                                                         %for Dirac brackets
\newcommand{\matrixel}[3]{\left< #1 \vphantom{#2#3} \right| #2 \left| #3 \vphantom{#1#2} \right>}
                                                         %for Dirac matrix elements
\newcommand{\grad}[1]{\gv{\nabla} #1}                    %for gradient
\let\divsymb=\div                                        %rename builtin command \div to \divsymb
\renewcommand{\div}[1]{\gv{\nabla} \cdot #1}             %for divergence
\newcommand{\curl}[1]{\gv{\nabla} \times #1}             %for curl
\newcommand{\sch}{Schr$\ddot{\rm o}$dinger }
\newcommand{\HRule}{\rule{\linewidth}{0.5mm}}

%-----------HYPERLINKS
\usepackage{hyperref}
\hypersetup{
  colorlinks,
  citecolor=black,
  filecolor=black,
  linkcolor=black,
  urlcolor=black
}