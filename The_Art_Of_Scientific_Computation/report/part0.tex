\chapter{Introduction}\label{part:into}

As we all know, natural selection is the progress that eliminates inferior species gradually over time and organisms today are products of it. In other words, the biological structures of livings are fitting for current enviroment.In the mean time, It is clearly obvious that most animals have two eyes (referred to as Binocular vision). So what advantages of this biological structure? Why it is the choice of nature rather than single eye? One possible reason is that having two eyes provides a much wider field of view. However, research shows that humans have a maximum horizontal field of view of 194 degrees with two eyes but 114 degrees of this is the binocular field of view (seen by both eyes)\cite{Howard1995Binocular}. Why so much overlap of each field of view rather than less overlap and wider field of uni-ocular fields (seen by only one eye)? 

The answer is having two eyes gives us perception of depth and 3-dimensional structure obtained on the basis of visual information deriving from them, which is referred to as stereopsis. Because the eyes of humans and many animals are located at different lateral positions on the head, binocular vision results in two slightly different images projected to the retinas of the eyes. The differences are mainly in horizontal position in two images, refer to binocular disparities. Our brain process these informations and gives precise depth perception.

In this project. We investigate how binocular works by implementing a stereo vision model. In stereo vision, two cameras instead of eyes are displaced horizontally from one another, and used to obtain two differing views on a scene, in a manner similar to human binocular vision. By comparing these two images, relative depth information can be extracted in the form of a disparity map, which encodes the difference in horizontal coordinates of corresponding image points. The values in this disparity map are inversely proportional to the scene depth at the corresponding pixel location.

Chapter 2 is about implementing spatial cross correlation computation model. In the first two sections, one dimensional cross correlation model is built, which is very useful in signal processing.  To test the effectiveness of it, we apply it to a particular signal offset compute task. In the next two sections, the model is extended to two dimensions, which can work in image comparison. Similarily, the model is applied to a particular pattern search task. Cause spatial cross correlation is time consuming,  a faster method to achieve same computational ability of cross correlation is needed. So we explore and implement spectral cross correlation in section 5, and compare the efficiency of two methods. The next section is an application of one dimensional cross correlation about pattern detection. 

In Chapter 3, the image comparison model and calibration model are built separately and combined to reconstruct a 3D scenery from two 2 calibrate images. In first two sections, calibration model is created by using calibration plants. In section 3, image comparison model is built and a mapping algorithm to extract depth data from comparison result is presented. Section 4 is about some optimization strategies to image comparison model, for speed or accuracy. Section 5 and 6 is the reconstruction of 3D scenery using models presented before. 

