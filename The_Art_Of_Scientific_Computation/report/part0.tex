\chapter{Introduction}\label{part:into}

As we all know, organisms today are products of natural selection, which eliminates inferior species gradually over time. In the mean time, It is clearly obvious that most animals have two eyes, referred to as Binocular vision. What advantages of this biological structure? One good reason is that having two eyes provides a much wider field of view. However, research shows that human humans have a maximum horizontal field of view of 188 degrees with two eyes but 124 degrees of this is the binocular field of view (seen by both eyes). Why so much overlap of each field of view rather than less overlap and wider field of uni-ocular fields (seen by only one eye)? 

The answer is having two eyes gives us perception of depth and 3-dimensional structure obtained on the basis of visual information deriving from them, which is referred to as stereopsis. Because the eyes of humans and many animals are located at different lateral positions on the head, binocular vision results in two slightly different images projected to the retinas of the eyes. The differences are mainly in horizontal position in two images, refer to binocular disparities. Our brain process these informations and gives precise depth perception.

In this project. We investigate how Binocular works by implementing a stereo vision model. In stereo vision, two cameras instead of eyes are displaced horizontally from one another, and used to obtain two differing views on a scene, in a manner similar to human binocular vision. By comparing these two images, we can extract relative depth information in the form of a disparity map, which encodes the difference in horizontal coordinates of corresponding image points. The values in this disparity map are inversely proportional to the scene depth at the corresponding pixel location.

In Chapter 2, we implement spatial cross correlation computation model. In the first two sections, we build a one dimension cross correlation model, which is very useful in signal processing and we apply it to a signal offset compute. In the next two sections, we extend our model to two dimensions, makes it work in image comparison and apply it to a particular pattern search task. Cause spatial cross correlation is time consuming, we want to find a faster method to achieve same computational ability of cross correlation, we explore and implement spectral cross correlation in section 5, and compare the run time of two methods. The next section is a application of 1d cross correlation, we apply this model to pattern detection. And the last section is our thinking and  future work about cross correlation.

In Chapter 3, the image comparison model is build and we illustrate the depth perception use the model. We first create a calibration model use calibration plant in first two sections. We then explore how to correct images for parallax and other distortions, cross calibrate images and use this information to create a mapping algorithm to extract depth data from two planar images in section 3,4,5, meanwhile, we develop some optimization strategies to the model, some of them are design for speed and some for accuracy improvement.

